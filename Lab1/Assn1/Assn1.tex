\documentclass{article}

\usepackage{amsmath}
\usepackage{algorithmic}
\usepackage{graphicx}
\usepackage{xspace}
\usepackage{multirow}

\begin{document}
\title{Hello \LaTeX \text{ }World}
\date{}
\author{\Large{Jucheol Moon}\\
Computer Engineering and Computer Science\\California State University Long Beach\\\texttt{jucheol.moon@csulb.edu\\}}
\maketitle




\begin{abstract}
This document is a model and instructions for \LaTeX \text{ } `article' class. 
\end{abstract}

\section{Introduction}
Welcome to the \LaTeX world.

\section{Ease of Use}

\subsection{Maintaining the Integrity of the Specifications}
The `article' class is used to format your paper and style the text. All margins, column widths, line spaces, and text fonts are prescribed.

\section{Styling Guide}

\subsection{Abbreviations and Acronyms}
Define abbreviations and acronyms the first time they are used in the text, 
even after they have been defined in the abstract.

\subsection{Equations}
\begin{equation}
\sum_{n=0}^{\infty}\frac{af^n}{n!}(x-a)^n
\label{Taylor}
\end{equation}
(\ref{Taylor}) is the famous Taylor series. Use ``(\ref{Taylor})'', not ``Eq. (\ref{Taylor})'' or ``equation (\ref{Taylor})'', except at the beginning of a sentence: ``Equation (\ref{Taylor}) is . . .''


Taylor series in a text would be $\sum_{n=0}^{\infty}\frac{af^n}{n!}(x-a)^n$.

\subsection{Lists}
Bullet style list.

\begin{itemize}
    \item item 1
    \item item 2
    \item item 3
\end{itemize}


Number style list.
\begin{enumerate}
    \item item 1
    \item item 2
    \item item 3
\end{enumerate}


\subsection{Figures and Tables}
\paragraph{Positioning Figures and Tables} Figure captions should be below the figures; table heads should appear above the tables. Insert figures and tables after they are cited in the text. Use the abbreviation ``Fig. \ref{fig: figure1}''. \\


\begin{table}[ht]
    \centering
    \caption{Table Type Styles}
    \begin{tabular}[t]{|c||c|l|l|}
    \hline
    \multirow{2}{*}{\begin{tabular}[c]{@{}c@{}}\textbf{Table} \\ \textbf{Head}\end{tabular}} & 
    \multicolumn{3}{c|}{\textbf{Table Column Head}} \\ \cline{2-4} & \textit{\textbf{Table column subhead}} & 
    \multicolumn{1}{c|}{\textit{\textbf{Subhead}}} & 
    \multicolumn{1}{c|}{\textit{\textbf{Subhead}}} \\ \hline
    \multicolumn{1}{|l||}{} & 
    \multicolumn{1}{l|}{} & & \\ \hline
    \end{tabular}
    \label{tab: table1}
\end{table}


\begin{figure}[ht]
    \centering
    \includegraphics[scale = 0.1]{fig1.png}
    \caption{Working Example}
    \label{fig: figure1}
\end{figure}



\subsection{Algorithms}
\begin{algorithmic}
\STATE $i\gets 10$
\IF {$i\geq 5$} 
        \STATE $i\gets i-1$
\ELSE
        \IF {$i\leq 3$}
                \STATE $i\gets i+2$
        \ENDIF
\ENDIF 
\end{algorithmic}


\subsection{Source codes}
\begin{verbatim}
public class HelloWorld{
    public static void main(String[] args) {
        System.out.println("Hello, World");
    }
}
\end{verbatim}

\subsection{References}

Please number citations consecutively within brackets \cite{Eason}. The sentence punctuation follows the bracket \cite{Maxwell}. Refer simply to the reference number, as in \cite{Jacobs}---do not use ``Ref. \cite{Jacobs}'' or ``reference \cite{Jacobs}'' except at the beginning of a sentence. 

\begin{thebibliography}{00}
\bibitem{Eason} G. Eason, B. Noble, and I. N. Sneddon, ``On certain integrals of Lipschitz-Hankel type involving products of Bessel functions,'' Phil. Trans. Roy. Soc. London, vol. A247, pp. 529--551, April 1955.
\bibitem{Maxwell} J. Clerk Maxwell, A Treatise on Electricity and Magnetism, 3rd ed., vol. 2. Oxford: Clarendon, 1892, pp.68--73.
\bibitem{Jacobs} I. S. Jacobs and C. P. Bean, ``Fine particles, thin films and exchange anisotropy,'' in Magnetism, vol. III, G. T. Rado and H. Suhl, Eds. New York: Academic, 1963, pp. 271--350
\end{thebibliography}

\end{document}
